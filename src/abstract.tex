\section{Abstract}

Das Ziel dieser Laborübung ist es mit dem Programm HyperChem die innere Energie (molare Energie) von Ethan und Tetrachlorethan zu bestimmen.
Diese molare Energie setzt sich aus 
\begin{align}
    \tilde{e} = e^{v} + e^{T} + e^{R} + e^{el}
\end{align}
\begin{table}[H]
    \centering
    \begin{tabular}{ll}
        $e^{v}$...Schwingungsenergie & $e^{T}$...Translationsenergie \\
        $e^{R}$...Rotationsenergie & $e^{el}$...Energie der elektrischen Anregung
    \end{tabular}
\end{table}
zusammen.\\
Wenn man ein Molekül auf 0K abkühlt, fällt die Rotationsenergie und die Translationsenergie auf 0, aber selbst am absolutem Nullpunkt hat ein
Molekül immer noch Energie in Form von $e^V$ und $e^{el}$ . Diese beiden Formen der Energien werden mit HyperChem bestimmt. \\
Für diese Bestimmung gibt es zwei verschiedenen Verfahren, einerseits die semiempirischen und andererseits die ab-initio Methoden.
Der Unterschied zwischen den semiempirischen und den ab-initio Methoden besteht darin, dass bei
den semiempirischen Methoden experimentell ermittelte Parameter mit in die Berechnung einfließen, aber bei den ab-initio Methoden lediglich die 
Ortskoordinaten der Atome und die physikalischen Konstanten in der Berechnung verwendung finden. Dabei fällt auf das es eine relativ große Zahl an
semiempirischen Methoden gibt. Diese Diversität entstand aus dem Drang heraus, für bestimmte Problemstellungen zu optimieren. Im Verlauf der Auswertung
wird sich zeigen das zwei dieser Methoden für unsere Fragestellung keine korrekten Ergebnisse liefern. \\
Sobald man dann die Nullpunktsenergien bestimmt hat, kann man über sie durch folgenden Zusammenhang,
\begin{align}
    \Delta \braket{\tilde{e}} (T) = \Delta e^V + \Delta e^{el} 
\end{align}
den Energieunterschied zwischen den verschiedenen Konstitutionen (gestaffelt oder verdeckt) bestimmen, da die temperaturabhängigen Beiträge der
Translation und Rotation sich gegenseitig aufheben. 
Eine besonders wichtige Energieform für die Chemie ist die Schwingungsenergie, da die Manipulation (Anregung) von Schwingungen der Moleküle durch
Licht die Grundlage der Spektroskopie bildet. Dabei geht das Molekül, welches eine Grundschwingung besitzt in ein Schwingungsmuster über, welches 
mehr Energie benötigt. Da man diese Anregung auch thermisch erreichen kann, kann man über folgende Beziehung den Energieunterschied zwischen 
angeregtem und nicht angeregtem Zustand berechnen.
\begin{align}
    \Delta E = k_b T
\end{align}
Da nicht alle Teilchen immer gleich viel Energie haben, sich also nie alle Moleküle in einem angeregtem Zustand befinden können ist es wichtig
die Verteilung zwischen den Zuständen zu kennen. Für diese statistische Berechnung wird oft die Bolzmann-Verteilung zugrunde gelegt.
\begin{align}
    n_i = \frac{N_i}{N_0} = e^{-\frac{\Delta E}{k_b T}}
\end{align} 

\begin{table}[H]
    \centering
    \begin{tabular}{ll}
        $n_i$...Verteilungsunterschied & $N_i$...Teilchen im angeregtem Zustand \\
        $N_0$...Teilchen im Grundzustand & $k_b$...Bolzmannkonstante
    \end{tabular}
\end{table}