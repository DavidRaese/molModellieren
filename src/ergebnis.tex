\newpage
\section{Zusammenfassung}

Bei Ethan wurden jeweils 8 maximal und 8 minimal Werte, einerseits für die Nullpunktsenergie und andererseits für die Energie der elektronischen
Anregung, mit 8 verschiedenen Methoden ermittelt. Diese Werte wurden dazu genutzt, um aus der maximalen und minimalen Nullpunktsenergie und der 
maximalen und minimalen Energie der elektronischen Anregung, die Energie zu ermitteln, die das Molekül benötigt, um vom staggered (gestaffelt) 
Konformer ins eclipsed (überlappend) Konformer überzugehen. \\
Unter den Werten befindet sich jedoch 1 Wert, der ZINDOS-Wert, welcher sehr auf einen Ausreißer hindeutet, jedoch handelt es sich hier um keinen, 
da der Dixon und Dean Ausreißertest die Hypothese, dass es sich um einen handelt, widerlegt. Der Ausreißertest ist nicht inkludiert im Protokoll, 
wurde jedoch durchgeführt. \\
Der Wert der Rotationsbarriere ergibt sich aus dem gemitteltem Wert der Rotationsbarrieren aller Methoden und weicht deshalb vom Literaturwert ab.
Die Rotationsbarriere für Ethan beträgt $1.449 \pm 0.484$ kcal/mol.\\
Da man für die Schwingungsspektren von Ethan in HyperChem Wellenzahlen raussuchen musste und dies leider nicht gemacht wurde in der Einheit,
mussten wir auf Werte von Kollegen/Kolleginnen zurückgreifen, um die Berechnung dann doch durchführen zu können. \\
Die Geschwindigkeitskonstante beträgt laut Berechnung $1.57 \cdot 10^{-12} s^{-1}$.
Für das Tetrachlorethan wurden auch jeweils 4 maximal und 4 minimal Werte, einmal für die Nullpunktsenergie und einmal für die Energie der elektronischen Anregung, mit 4 verschiedenen Methoden ermittelt. Da die 4 Chloratome viel massiger sind als die 4 Wasserstoffatome die sie ersetzen, erhält man hier eine viel höhere Rotationsbarriere, da die überlappenden Chloratome träger und vorallem größer sind, als die Wasserstoffatome. Durch ihre Größe beeinflussen sie sich viel mehr im eclipsed Konformer und sorgen damit für einen höheren benötigten Energiebetrag, um aus dem staggered Konformer ins eclipsed Konformer überzugehen, als Ethan.
Die Rotationsbarriere für Tetrachlorethan beträgt $9.282 \pm 0.749$ kcal/mol.

%----------------------------------------------------------------------------------------------------

